\chapter{Conjuntos difusos}
El contenido de este capítulo se basa en las definiciones que podemos encontrar en la referencia \cite{fuzzyintro}

\section{Subconjuntos difusos}
La teoría clásica de conjuntos sólo abarca la posibilidad de que un elemento pertenezca o no, a un conjunto. Pero la realidad no es así, y pueden existir ciertos casos en lo que la \textbf{pertenencia} o no, a un conjunto haya que definirlo mediante un \textbf{grado de pertenencia.}

\subsection{Subconjuntos difusos}
En primer lugar, introduciremos el concepto de conjunto difusos que nos servirá para formalizar el concepto de conjuntos con grados de pertenencia.

\begin{definicion}[Subconjunto difuso]
	\label{def:subconjunto_difuso}
	Un \textbf{subconjunto difuso} $A$ es un par ordenado $(\mu_A, \mathbb{U})$ con:
	\[
		\mu_A : \mathbb{U} \longrightarrow [0,1]
	\]
	Denominamos $\mu_A$ \textbf{función de pertenencia.}
\end{definicion}
Tenemos que notar que esta función de pertenencia no define necesariamente una probabilidad, y no hace referencia por ejemplo a la probabilidad de que una persona sea alta o baja, si no que nos da una medida de cuanto de alta o baja es.\\

Por tanto, es natural definir la igualdad de conjuntos de la siguiente forma:

\begin{definicion}[Igualdad de conjuntos difusos]
	\label{def:igualdad}
	Decimos que dos conjuntos difusos $A$ y $B$ en $\mathbb{U}$ son iguales si para todo $x \in \mathbb{U}$, se cumple $\mu_A(x) = \mu_B(x)$
\end{definicion}

Si se que cumpliese que $\mu_A(\mathbb{U})=\{0, 1\}$ \textbf{tendríamos que $A$ es un conjunto clásico}, y llamamos a $\mu_A$ función característica de $A$. En este caso, tenemos que si $x \in A$, entonces $\mu_A(x)=1$, y por el contrario si $x \notin A$ entonces, $\mu_A(x)=0$

Por tanto, \textbf{la función $\mu_A$ representa una generalización del concepto de función característica clásica} dónde $\mu_A$ representa el grado de pertenencia a un conjunto.

\subsection{$\alpha$-corte}
Ahora introducimos un concepto fundamental en la teoría de conjuntos difusos, y es el concepto de $\alpha$-corte, que nos permitirá crear particiones de los conjuntos separados por los valores de la función de pertenencia.
\begin{definicion}[$\alpha$-corte]
	\label{def:alpha_corte}
	Dado un conjunto difuso $A$, los $\alpha$-corte son los subconjuntos clásicos dados por:
	\[
		[A]_\alpha = \left\{
			\begin{array}{ccc}
			\{x \in \mathbb{U} : \mu_A(x) \geq \alpha \} & si & \alpha \in (0, 1] \\
			cl\{x \in \mathbb{U} : \mu_A(x) > 0\} & si & \alpha=0
			\end{array}
		\right.
	\]
	Además, se define:
	\[
		soporte ~ A = \{x \in \mathbb{U} : \mu_A(x) > 0 \}
	\]
	\[
		nucleo ~ A = \{x \in \mathbb{U} : \mu_A(x) = 1 \}
	\]
\end{definicion}
De esta definición, se puede extraer que un conjunto difuso también puede estar definido por sus $\alpha$-corte, de manera que \textbf{dos conjuntos difusos $A$ y $B$ son iguales, si todos sus $\alpha$-corte son iguales.} 

Dado unos $\alpha$-corte podemos construir una función de pertenencia de la siguiente forma: \cite{apuntesfuzzy}

\[
	\mu_A = \max{\{\alpha A_\alpha(x) : \alpha \in [0, 1]}\}
\]
\[
	A_\alpha(x) = \left\{
		\begin{array}{ccc}
			1 & si & x \in [A]_\alpha \\
			0 & si & x \notin [A]_\alpha
		\end{array}
	\right.
\]

Desde aquí, centraremos nuestro estudio en los $\alpha$-corte de los conjuntos difusos.

\section{Números difusos}
Para poder trabajar con sistemas de ecuaciones diferenciales difusos, es interesante introducir el concepto de números difusos, que podrían ser valores iniciales de nuestro sistema, o números que aparecen directamente en nuestro sistema de ecuaciones diferenciales. \\
Necesitamos en primer lugar dos definiciones antes de definir el concepto de número difuso.

\begin{definicion}[Conjunto difuso normal]
	\label{def:difuso_normal}
	Un conjunto difuso $A$ decimos que es normal si $nucleo ~ A \neq \emptyset$
\end{definicion}

\begin{definicion}[Conjunto difuso convexo]
	\label{def:difuso_convexo}
	Dado un conjunto difuso $A$ decimos que es convexo si su función de pertenencia es cuasicóncava, esto es:
	\[
		\mu_A(\lambda x + (1-\lambda)y) \geq \min{\{\mu_A(x), \mu_A(y)\}}, \lambda \in [0, 1], x, y \in \mathbb{U}
	\]
\end{definicion}

En este caso, si $\mathbb{U}=\mathbb{R}$ tendríamos que si $A$ es un conjunto difuso convexo, entonces los $\alpha$-corte son intervalos.

Finalmente, introducimos el concepto de número difuso:

\begin{definicion}[Número difuso]
	\label{def:numero_difuso}
	Decimos que $A$ es un número difuso si $\mathbb{U}=\mathbb{R}$, $A$ es normal y convexo, y además su función de pertenencia es continua por la derecha.
\end{definicion}


Observamos que si $x \in \mathbb{R}$ el conjunto $\{x\}$ es un número difuso, con la función de pertenencia.

%\begin{figure}[h]
%	\centering
%	\includegraphics[width=0.6\textwidth]{numero_difuso}
%	\caption{Ejemplo de número difuso}
%	\label{fig:numero_difuso}
%\end{figure}

\subsection{Caracterización números difusos}
A continuación, introducimos dos teoremas que nos permitirán caracterizar los números difusos. Las demostraciones de estos dos resultados se pueden encontrar \cite{apuntesfuzzy}.

\begin{teorema}[Teorema de Stacking]
	Sea $A$ un número difuso, entonces:
	\begin{enumerate}
		\item Sus $\alpha$-cortes son intervalos cerrados no vacíos para todo $\alpha \in [0, 1]$
		\item Si $0 \leq \alpha_1 \leq \alpha_2 \leq 1$ entonces $[A]_{\alpha_1} \subset [A]_{\alpha_2}$
		\item Para toda sucesión no decreciente de $\alpha_n \in [0, 1]$ tal que $a_n \longrightarrow \alpha$ se tiene que:
		\[
			\bigcap^\infty_{n=1} [A]_{\alpha_n} = [A]_\alpha
		\]
		\item Para toda sucesión no creciente $\alpha_n \in [0, 1]$ convergente a $0$ se tiene:
		\[
			cl\left(
				\bigcup^\infty_{n=1} [A]_{\alpha_n} = [A]_0
			\right)
		\]
	\end{enumerate}
\end{teorema}

\begin{teorema}[Teorema de caracterización]
	Sea $A=\{A_\alpha : \alpha \in [0, 1]\}$ una familia de subconjuntos de $\mathbb{R}$ tal que:
	
	\begin{enumerate}
		\item Sus $\alpha$-cortes son intervalos cerrados no vacíos para todo $\alpha \in [0, 1]$
		\item Si $0 \leq \alpha_1 \leq \alpha_2 \leq 1$ entonces $[A]_{\alpha_1} \subset [A]_{\alpha_2}$
		\item Para toda sucesión no decreciente $\alpha_n \in [0, 1]$ tal que $\alpha_n \longrightarrow \alpha$ se tiene que
		$$
		\bigcap^\infty_{n=1} [A]_{\alpha_n}=[A]_\alpha
		$$
		\item Para toda sucesión no creciente $\alpha_n \in [0, 1]$ convergente a $0$ se tiene:
		$$
		cl\left(\bigcup^\infty_{n=1} [A]_{\alpha_n}\right)=[A]_0
		$$
	\end{enumerate}
	Entonces, $A$ es un número difuso.
\end{teorema}


\begin{ejemplo}
	Sea $\IU=\IR$ y consideremos $\FUZZYM: \IR \longrightarrow [0, 1]$ definida de la siguiente forma:
	$$
	\FUZZYM(x)=\left\{
	\begin{array}{c c c}
	\frac{x-a}{b-a} & si & x \in [a, b] \\
	\frac{c-x}{c-b} & si & x \in (b, c] \\
	0, & si & x \notin [a, c]
	\end{array}
	\right.
	$$
	Con $a < b < c$. Decimos que \textbf{un número difuso definido de la forma anterior es triangular (a; b; c).} 
	
\end{ejemplo}


\section{Principio de extensión de Zadeh}
Nos gustaría ahora que dado un conjunto difuso, y una función clásica, pudiéramos obtener una función difusa, para esto, tenemos el principio de extensión de Zadeh.

\begin{definicion}[Principio de extensión de Zadeh]
	Sean $\mathbb{U}$ y $\mathbb{V}$ dos conjuntos de universos, y sea $f: \mathbb{U} \longrightarrow \mathbb{V}$ una función clásica. Definimos el principio de extensión de Zadeh para todo conjunto difuso $A$ en $\mathbb{U}$ con $\mu_A$ su función de pertenencia, de manera que $\hat{f}(A)$ en $\mathbb{V}$ y su función de pertenencia viene dada por:
	
	$$
		\mu_{\hat{f}(A)}=\left\{
			\begin{array}{ccc}
				\sup_{x\in f^{-1}(y)} \mu_A(x) & si & f^{-1}(y)\neq\emptyset\\
				0 & si & f^{-1}=\emptyset
			\end{array}
		\right.
	$$
\end{definicion}


Notemos, que si la función es inyectiva, la función de pertenencia se simplificaría:
$$
	\mu_{\hat{f}(A)}=\left\{
		\begin{array}{ccc}
			\mu_A(f^{-1}(y)) & si & f^{-1}(y)\neq\emptyset\\
			0 & si & f^{-1}=\emptyset
		\end{array}
	\right.
$$
En el apéndice, se pueden encontrar ejemplos de funciones de pertenencias dadas por el principio de extensión de Zadeh.

\subsection{Teoremas de continuidad}
Sería ideal que no importase el orden el que obtenemos los $\alpha$-corte de la imagen de una función mediante el principio de extensión de Zadeh, y esto lo asegura el siguiente teorema.

\begin{teorema}
	Sea $f : \mathbb{R}^n \longrightarrow \mathbb{R}^m$ una función.
	\begin{enumerate}
		\item Si $f$ es sobreyectiva, entonces $[\hat{f}(A)]_\alpha = f([A]_\alpha)$ si y sólo si $\sup\{\mu_A(x) : x \in f^{-1}(y)\}$ para todo $y \in \mathbb{R}^m$
		\item Si $f$ es continua, entonces $\hat{f} : \mathcal{F}_\mathcal{H}(\mathbb{R}^n) \longrightarrow \mathcal{F}_\mathcal{H}(\mathbb{R}^n)$ está bien definido, y además,  
		$$[\hat{f}(A)]_\alpha = f([A]_\alpha)$$
		para todo $\alpha \in [0, 1]$
	\end{enumerate}
\end{teorema}

La primera implicación es clara por la definición de principio de extensión de Zadeh, para la segunda implicación veamos un caso más general:


\begin{teorema}
	Sean $\mathbb{U}$ y $\mathbb{V}$ unos espacios de Haussdorf, y sea $f: \mathbb{U} \longrightarrow \mathbb{V}$ una función. Si $f$ es continua, entonces  $\hat{f} : \mathcal{F}_\mathcal{H}(\IR^n) \longrightarrow \mathcal{F}_\mathcal{H}(\IR^n)$ está bien definida y $$[\hat{f}(A)]_\alpha = f([A]_\alpha)$$
	para todo $\alpha \in [0, 1]$
\end{teorema}

\begin{proof}
	Por la definición del principio de Zadeh, tenemos que $\hat{f}(A)$ es un subconjunto difuso de $\mathbb{V}$. \\
	Para probar que $\hat{f} : \mathcal{F}_\mathcal{H}(\IR^n) \longrightarrow \mathcal{F}_\mathcal{H}(\IR^n)$ hay que ver que todos los $\alpha-corte$ $[\hat{f}(A)]_\alpha$ son no vacíos, compactos en $\mathbb{V}$. \\
	Dado que $f$ es continua por hipótesis, la imagen de compactos, son compactos, por tanto, sólo debemos probar $[\hat{f}(A)]_\alpha = f([A]_\alpha)$. \\
	Probémoslo por doble inclusión.
	
	\begin{itemize}
		\item $[\hat{f}(A)]_\alpha \subset f([A]_\alpha)$. Sea $\mathcal{A} \in \mathcal{F}_\mathcal{H}(\IU)$ y $y \in f([A]_\alpha)$. Por tanto, existe al menos un $x \in [A]_\alpha$ tal que $f(x)=y$. Por el principio de extensión de Zadeh, tenemos $\mu_{\hat{f}(A)}(y)=\sup_{x\in f^{-1}(y)} \mu_A(x) \geq \alpha$. De donde, $y \in [\hat{f}(\mathcal{A})_\alpha]$
		\item Por otro lado, veamos que $[\hat{f}(A)]_\alpha \supset f([A]_\alpha)$. Dado que $\mathbb{V}$ y $\mathbb{U}$ son espacios de Haussdorf, un punto $y \in \mathbb{V}$ es cerrado. Y además, dado que $f$ es continua, $f^{-1}(y)$ es cerrado. Dado que $[A]_0$ es compacto, ya que es la clausura, la intersección de compactos también es compactos, por tanto $f^{-1}(y) \cap [A]_0$ es compacto. Para $\alpha>0$, consideramos $y \in [\hat{f}(A)]_\alpha$. Entonces $\mu_{\hat{f}(A)}(y)=\sup_{x\in f^{-1}(y)} \mu_A(x) \geq \alpha>0$, y además, existe un $x\in f^{-1}(y)$ tal que $f^{-1}(y) \cap [A]_0 \neq \emptyset$
	\end{itemize}
	Finalmente, debido a que $\mu_A(x)$ es continúa por la derecha, y $f^{-1}(y) \cap [A]_0$ es compacto, existe un $x \in f^{-1}(y) \cap [A]_0$ con $\mu_{\hat{f}(\mathcal{A})}(y)=\mu_{\mathcal{A}}(x) \geq \alpha$. Esto es porque $y=f(x)$ para algún $x \in [A]_\alpha$.
	
	Para $\alpha=0$, obtenemos:
	
	$$
	\bigcup_{\alpha \in (0, 1]} [\hat{f}(\mathcal{A})]_\alpha = \bigcup_{\alpha \in (0, 1]} f([\mathcal{A}]_\alpha) \subset f([\mathcal{A}]_0).
	$$
	
	Dado que $f([\mathcal{A}]_0)$ es cerrado:
	
	$$
	[\hat{f}(\mathcal{A})]_0 = cl\left( \bigcup_{\alpha \in (0, 1]} [\hat{f}(\mathcal{A})]_\alpha  \right) = cl\left(\bigcup_{\alpha \in (0, 1]} f([\mathcal{A}]_\alpha) \right)	\subset f([\mathcal{A}]_0).
	$$
	
	Y por la doble inclusión anterior, tenemos que $[\hat{f}(\mathcal{A})]_\alpha \subset f([\mathcal{A}]_\alpha)$ para todo $\alpha \in [0, 1]$
\end{proof}


\section{Aritmética difusa}
El siguiente paso \textbf{para poder construir métodos numéricos es necesario definir las operaciones aritméticas básicas} entre conjuntos difusos.\\
La definiciones de \textbf{estas operaciones son bastante naturales}, pero pueden ocasionarnos algunos problemas en ciertos escenarios.\\
Debido a la equivalencia entre trabajar con conjuntos difusos, y sus $\alpha-corte$, daremos todas las operaciones en términos de $\alpha-cortes$.\\
En primer lugar, \textbf{recordemos la definición habitual de aritmética en teoría de conjuntos clásicos.}

\subsection{Aritmética en conjuntos clásicos}
Sean $A, B$ dos conjuntos entonces:
\begin{itemize}
	\item $A+B=\{a+b : a \in A, b\in B\}$
	\item $A - B =\{a - b : a \in A, b\in B\}$
	\item $A * B =\{ab : a \in A, b\in B\}$
	\item $A / B =\{a/b : a \in A, b\in B\}$
\end{itemize}

Recordados los conceptos básicos de aritmética en conjuntos clásicos, pasamos a generalizarlo para conjuntos difusos.

\subsection{Aritmética en conjuntos difusos}
Sean $\mu_A, \mu_B$ dos funciones de pertenencia y sea $\odot \in \{+, -, \cdot, \div\}$ se define la función de pertenencia de la operación aritmética como:
$$
\mu_{A \odot B}(x) = \sup_{a \odot b = c} \min\{\mu_A(a), \mu_B(b)\}
$$
Y dado que \textbf{las operaciones aritméticas son funciones continuas}, es equivalente trabajar con los $\alpha-corte$, \textbf{aplicando el principio de extensión de Zadeh tenemos:}

Sean $A$ y $B$ dos números difusos con $\alpha-corte$ dados por $[A]_\alpha=[a_\alpha^-, a_\alpha^+]$ y $[B]_\alpha=[b_\alpha^-, b_\alpha^+]$, podemos definir entonces las operaciones aritméticas como:

$$
[A+B]_\alpha = [a_\alpha^- + b_\alpha^-, a_\alpha^+ + b_\alpha^+]
$$

$$
[A-B]_\alpha = [a_\alpha^- - a_\alpha^+, a_\alpha^+ - b_\alpha^-]
$$

$$
[A \cdot B]_\alpha = \left[ \min_{s, r \in \{-, +\}} a_\alpha^s \cdot b_\alpha^r, \max_{s, r \in \{-, +\}} a_\alpha^s \cdot b_\alpha^r\right]
$$

$$
[A \div B]_\alpha = \left[ \min_{s, r \in \{-, +\}} \frac{a_\alpha^s}{b_\alpha^r}, \max_{s, r \in \{-, +\}} \frac{a_\alpha^s}{b_\alpha^r}\right]
$$

\subsubsection{Problemas al definir estas operaciones aritméticas}
Nuestro objetivo final es \textbf{definir la diferencial de una función}, para funciones escalares de una sola variable podemos definir la diferencial como:
$$
\lim\limits_{h\rightarrow 0^+} \frac{f(x+h) - f(x)}{h}
$$
Consideremos ahora una función $f(x)=A\in \mathcal{F}_\mathcal{H}(\IU)$, donde $A$ es un número difuso constante. \\
$$f(x+h) - f(x)=[A-A]$$
Sean $[A]_\alpha$ los $\alpha-cortes$ de $A$ entonces:
$$
[A-A]_\alpha = [a_\alpha^- - a_\alpha^+, a_\alpha^+ - a_\alpha^-]
$$
Si $A\neq 0$ tenemos que $[A-A]_\alpha \neq 0$, por tanto al dividir por $h \longrightarrow 0$, tenemos una indeterminación, de donde, tal y \textbf{como hemos definido las operaciones aritméticas para los conjuntos difusos no tendríamos definidas las derivadas de funciones constantes}, y esto, es un problema. \\
Necesitamos definir \textbf{un nuevo concepto de diferencia}, que para ello, \textbf{utilizaremos la diferencia de Hukuhara.}

\subsection{Hukuhara y diferencia generalizada} \label{def:hukukara}
Debido al problema especificado en la sección anterior, \textbf{necesitamos definir una operación resta que cumpla que $A-A=\{0\}$}, Hukuhara dio una definición de resta que cumplía esta propiedad.

\begin{definicion}
	Dados dos números difusos $A, B \in \mathcal{F}_\mathcal{C}\IR$ la \textbf{diferencia de Hukuhara (H-Diferencia)} se define como $A \circleddash_H B = C$ donde $C$ es el número difuso que cumple $A=B+C$, si existe.
\end{definicion}

\begin{observacion}
	Sean $A, B, C \in \mathcal{F}_\mathcal{C}\IR$ consideremos sus $\alpha - cortes$, por tanto $[A]_\alpha = [a^-_\alpha, a^+_\alpha]$, $[B]_\alpha = [b^-_\alpha, b^+_\alpha]$ y $[C]_\alpha = [c^-_\alpha, c^+_\alpha]$. \\
	De donde,
	$$
	[a^-_\alpha, a^+_\alpha] = [b^-_\alpha + c^-_\alpha, b^+_\alpha + c^+_\alpha]
	$$
	Por tanto, 
	$$
	[A \circleddash_H B]_\alpha = [a_\alpha^- - b_\alpha^-, a_\alpha^+ - b_\alpha^+]
	$$
\end{observacion}
Con esta observación es fácil ver que la H-Diferencia cumple que $A-A=\{0\}$


\section{Interactividad}
El principio de extensión de Zadeh se puede aplicar a funciones de distinto número de argumentos. Los ejemplos más simples podrían ser la suma, la resta, la multiplicación y la división de números difusos. Esta situación es más compleja, debido a que tenemos que tener en cuenta las estructuras entre los distintos argumentos. Esta dependencia mutua entre los distintos conjuntos difusos, viene dada por una función de pertenencia común que llamaremos \textbf{función de pertenencia conjunta}. En términos de conjuntos difusos, esta dependencia se llama interactividad.

\begin{definicion}[Interactividad, función de pertenencia conjunta]
	Sea $\hat{a} \in \mathcal{F}(V)$ y sea $\hat{b} \in \mathcal{F}(W)$. Entonces la interactividad de $\hat{a}$ y $\hat{b}$ se define por la función de pertenencia conjunta: $\mu_{\hat{a}, \hat{b}} : V \times W \rightarrow [0, 1]$
\end{definicion}
Para calcular las funciones de pertenencia marginales, simplemente tenemos que aplicar el principio de extensión de Zadeh:

\begin{definicion}[Función marginal de una función de pertencia conjunta]
	Definimos la función de pertenencia marginal respecto $a$ como:
	\[
	\mu_a(a) = \sup\lim\limits_{b \in W} \mu_{\hat{a}, \hat{b}} (a, b)
	\]
\end{definicion}

Se suele suponer no interactividad al trabajar con varios conjuntos difusos, esto es;

\begin{definicion}[No interactivos o independientes]
	Dos conjuntos difusos $\hat{a} \in \mathcal{F}(V)$ y $\hat{b} \in \mathcal{F}(W)$ se dicen que son no interactivos, o independientes si $\mu_{\hat{a}, \hat{b}} = \min(\mu_{\hat{a}}(a), \mu_{\hat{b}}(b))$
\end{definicion}

\begin{ejemplo}
En el caso de conjuntos no interactivos, podemos escribir las operaciones aritméticas de los conjuntos difusos de la siguiente manera:

\[
[\hat{a} + \hat{b}]_\alpha = [[\hat{a}^-_\alpha + [\hat{b}]_\alpha^-, [\hat{a}]^+_\alpha + [\hat{b}]_\alpha^+]
\]

\[
[\hat{a} - \hat{b}]_\alpha = [[\hat{a}^-_\alpha - [\hat{b}]_\alpha^-, [\hat{a}]^+_\alpha - [\hat{b}]_\alpha^+]
\]

\[
[\hat{a} * \hat{b}]_\alpha = \max\{ [\hat{a}]_\alpha^i [\hat{b}]^j_\alpha \}, i, j \in \{+, -\}
\]

\[
[\hat{a} / \hat{b}]_\alpha = \max\{ [\hat{a}]_\alpha^i / [\hat{b}]^j_\alpha \}, i, j \in \{+, -\}
\]
\end{ejemplo}

\section{Métrica en conjuntos difusos}
En esta sección vamos a generalizar la definición de espacio métrico a conjuntos difusos, y vamos a dar algunos resultados importantes. Todas las demostraciones de esta sección pueden encontrarse en \cite{apuntesfuzzy}. \\
En primer lugar, nos acercaremos al concepto de métrica:

\begin{definicion}[Pseudométrica]
	Sean $A$ y $B$ dos subconjuntos de un espacio métrico $\IU$ compactos. Entonces definimos la pseudométrica como:
	\[
		\rho(A, B) = \sup\limits_{a \in A} d(a, B)
	\]
	Donde:
	\[
		d(a, B) = \inf\limits_{b \in B} ||a-b||
	\]
	es la separación de Haussdorf
\end{definicion}

\begin{definicion}[Métrica de Pompeiu-Hausdorff]
	\label{def:metricadifusa}
	Sean $A$ y $B$ dos conjuntos difusos en $\IU$, un espacio métrico. La métrica de Pompeiu-Hausdorff, denotada por $d_\infty$ se define:
	\[
		d_\infty(A, B) = \sup\limits_{\alpha \in [0, 1]} \max\{\rho([A]_\alpha, [B]_\alpha), \rho([B]_\alpha,  [A]_\alpha)\}
	\]
	
	Si $A$ y $B$ fueran números difusos, tendríamos:
	\[
		d_\infty(A, B) = \sup\limits_{\alpha \in [0, 1]} \max\{|a_\alpha^- - b_\alpha^-|, |a_\alpha^+ - b_\alpha^+|\}
	\]
\end{definicion}

\begin{teorema}[\cite{banachfuzzy}]
	\label{teorema:banach}
	El espacio de los números difusos, con la métrica $d_\infty$ es un espacio de Banach.
\end{teorema}

\section{Funciones difusas}
En la literatura sobre funciones difusas existen dos definiciones para referirnos a funciones difusas, originalmente Dubois y Prade (1980) definieron los siguientes conceptos:

\begin{definicion}[Funciones definidas en conjuntos difusos]
	\label{def:fizzusetvaluedfunc} Decimos que una función \textbf{$F$ es una función definida en un conjunto difuso} si:
	\begin{enumerate}
		\item $dom ~ F \subset \IR$
		\item $Im ~ F \subset \mathcal{F}_\mathcal{H}(\IR^n)$
	\end{enumerate}
\end{definicion}

\begin{ejemplo}
	La función $f(x) = A x$ donde $A=[a, b]$ con $a, b \in \IR$, es una función definida en conjuntos difusos. Sus imágenes son intervalos.
\end{ejemplo}

\begin{ejemplo}
	La función $f(x) = A x$ donde $A=(a;b;c)$ con $a<b<c$, es una función definida en conjuntos difusos. Además, su imágenes son conjuntos números triangulares.
\end{ejemplo}

\iffalse
\begin{definicion}[Ristra de funciones difusas]
	Definimos una ristra de funciones difusas como el subconjunto difuso del espacio de funciones. \\
	Para cada ristra de funciones difusas, podemos darle una función definida en un conjunto difuso. Para cada $F \in \mathcal{F}(E(I; \IR^n))$ donde $E(I, \IR^n)$ es una espacio de funciones de $I \subset \IR$ a $\IR^n$. Donde:
	\[
		[F(t)]_\alpha = [F]_\alpha(t) = \{f(t) : f \in [F]_\alpha\}
	\]
\end{definicion}

\begin{ejemplo}
	Consideramos $f_1, f_2$ y $f_3$ funciones continuas en el intervalo $A=[a, b]$. Y definimos el conjunto difuso $F \in \mathcal{F}(C([a, b]; \IR))$ tal que:
	\[
		\mu_F(f) = \left\{
			\begin{array}{ccc}
				\alpha & si & f = f_1 + \alpha(f_2 - f_1) \\
				\alpha & si & f = f_3 + \alpha(f_2 - f_3) \\
				0 & otro~caso & 
			\end{array}
		\right.
	\]
	
	Entonces, decimos que $F$ es una ristra de funciones difusas.
\end{ejemplo}

\fi

\subsection{Continuidad de funciones difusas}
Introduciremos en primer lugar dos conceptos de continuidad sobre funciones reales que toman valores en subconjuntos reales. Y luego, usaremos esta misma idea para definir la continuidad sobre conjuntos difusos

\begin{definicion}[Función sobre conjuntos continua]
	Sea $F : \Omega \rightarrow \mathcal{P}(\IR^n)$, $\Omega \subset \IR^m$ decimos que $F$ es \textbf{semicontinua superiormente} en $t_0 \in \Omega$, si para todo $\varepsilon>0$ existe un $\delta>0$ tal que:
	\[
		\rho(F(t), F(t_0))<\varepsilon
	\]
	Si $||t-t_0||<\delta$ para $t \in \Omega$. \\
	Por otro lado, decimos que $F$  es \textbf{semicontinua inferiormente} en $t_0 \in \Omega$, si para todo $\varepsilon>0$ existe un $\delta>0$ tal que:
	\[
	\rho(F(t_0), F(t))<\varepsilon
	\]
	Si $||t-t_0||<\delta$ para $t \in \Omega$. \\
	Si una función es semicontinua superiormente e inferiormente, decimos que es \textbf{continua.}
\end{definicion}

\begin{definicion}[Función difusa continua]
	Sea $F : \Omega \rightarrow \mathcal{F}_\mathcal{H}(\IR^n)$, $\Omega \subset \IR^m$ decimos que $F$ es \textbf{semicontinua superiormente} en $t_0 \in \Omega$, si para todo $\varepsilon>0$ existe un $\delta>0$ tal que:
	\[
	\rho([F(t)]^\alpha, [F(t_0)]^\alpha)<\varepsilon
	\]
	Si $||t-t_0||<\delta$ para $t \in \Omega$, para todo $\alpha \in [0, 1]$. \\
	Por otro lado, decimos que $F$ es \textbf{semicontinua superiormente} en $t_0 \in \Omega$, si para todo $\varepsilon>0$ existe un $\delta>0$ tal que:
	\[
	\rho([F(t_0)]^\alpha, [F(t)]^\alpha)<\varepsilon
	\]
	Si $||t-t_0||<\delta$ para $t \in \Omega$, para todo $\alpha \in [0, 1]$. \\
	Si una función es semicontinua superiormente e inferiormente, decimos que es \textbf{continua.}
\end{definicion}

\subsection{Representación gráfica de funciones difusas}
En este sección se va a buscar una solución al problema de representar funciones difusas.

Una representación gráfica de una función difusa podemos desear que cumpla lo siguiente:
\begin{enumerate}
	\item El grado de pertenencia debe verse de forma intuitiva.
	\item Si nuestra función toma valores en $\IR^n$ y va $\mathcal{F}_\mathcal{H}(\IR^m)$, la gráfica debería de estar representada en $n+m+1$. Necesitamos una dimensión más debido a que los $\alpha$-corte son $m$-intervalos.
\end{enumerate}

\subsubsection{Gráfica de una función difusa}

Siguiendo los puntos de la introducción, vamos a tratar de dar una solución a cada uno de los puntos:

\begin{enumerate}
	\item El grado de pertenencia podemos representarlo usando colores. Fijamos un color inicial, que definiremos como $0$ y fijamos otro color final que definiremos como $1$. Todos los colores que hay entre el color inicial y final, representarán los grados de pertenencia. Por ejemplo, si tomamos como $0$ el color blanco $(255, 255, 255)$ y tomamos el color negro $(0, 0, 0)$ como color final, si queremos representar un valor con grado de pertenencia $\alpha$, deberíamos tomar el color: $(255(1-\alpha), 255(1-\alpha), 255(1-\alpha))$
	
	\item Por cada $x$ representamos una recta que contenga todos los valores que puede recorrer $f(x)$ y estos valores los pintamos según el grado de pertenencia especificado en el apartado anterior.
\end{enumerate}