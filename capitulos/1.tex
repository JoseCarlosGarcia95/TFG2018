\chapter{Conjuntos difusos}
El contenido de este capítulo se basa en las definiciones que podemos encontrar en la referencia \cite{fuzzyintro}

\section{Subconjuntos difusos}
La teoría clásica de conjuntos sólo abarca la posibilidad de que un elemento pertenezca o no, a un conjunto. Pero la realidad no es así, y pueden existir ciertos casos en lo que la \textbf{pertenencia} o no, a un conjunto haya que definirlo mediante un \textbf{grado de pertenencia.}

\subsection{Subconjuntos difusos}
En primer lugar, introduciremos el concepto de conjunto difusos que nos servirá para formalizar el concepto de conjuntos con grados de pertenencia.

\begin{definicion}[Subconjunto difuso]
	Un \textbf{subconjunto difuso} $A$ es un par ordenado $(\mu_A, \mathbb{U})$ con:
	\[
		\mu_A : \mathbb{U} \longrightarrow [0,1]
	\]
	Denominamos $\mu_A$ \textbf{función de pertenencia.}
\end{definicion}

Por tanto, es natural definir la igualdad de conjuntos de la siguiente forma:

\begin{definicion}[Igualdad de conjuntos difusos]
	Decimos que dos conjuntos difusos $A$ y $B$ en $\mathbb{U}$ son iguales si para todo $x \in \mathbb{U}$, se cumple $\mu_A(x) = \mu_B(x)$
\end{definicion}

Si se que cumpliese que $\mu_A(\mathbb{U})=\{0, 1\}$ \textbf{tendríamos que $A$ es un conjunto clásico}, y llamamos a $\mu_A$ función característica de $A$. En este caso, tenemos que si $x \in A$, entonces $\mu_A(x)=1$, y por el contrario si $x \notin A$ entonces, $\mu_A(x)=0$

Por tanto, \textbf{la función $\mu_A$ representa una generalización del concepto de función característica clásica} dónde $\mu_A$ representa el grado de pertenencia a un conjunto.

\subsection{$\alpha$-corte}
Ahora introducimos un concepto fundamental en la teoría de conjuntos difusos, y es el concepto de $\alpha$-corte, que nos permitirá crear particiones de los conjuntos separados por los valores de la función de pertenencia.
\begin{definicion}[$\alpha$-corte]
	Dado un conjunto difuso $A$, los $\alpha$-corte son los subconjuntos clásicos dados por:
	\[
		[A]_\alpha = \left\{
			\begin{array}{ccc}
			\{x \in \mathbb{U} : \mu_A(x) \geq \alpha \} & si & \alpha \in (0, 1] \\
			cl\{x \in \mathbb{U} : \mu_A(x) > 0\} & si & \alpha=0
			\end{array}
		\right.
	\]
	Además, se define:
	\[
		soporte ~ A = \{x \in \mathbb{U} : \mu_A(x) > 0 \}
	\]
	\[
		nucleo ~ A = \{x \in \mathbb{U} : \mu_A(x) = 1 \}
	\]
\end{definicion}
De esta definición, se puede extraer que un conjunto difuso también puede estar definido por sus $\alpha$-corte, de manera que \textbf{dos conjuntos difusos $A$ y $B$ son iguales, si todos sus $\alpha$-corte son iguales.} 

Dado unos $\alpha$-corte podemos construir una función de pertenencia de la siguiente forma: \cite{apuntesfuzzy}

\[
	\mu_A = \max{\{\alpha A_\alpha(x) : \alpha \in [0, 1]}\}
\]
\[
	A_\alpha(x) = \left\{
		\begin{array}{ccc}
			1 & si & x \in [A]_\alpha \\
			0 & si & x \notin [A]_\alpha
		\end{array}
	\right.
\]

Desde aquí, centraremos nuestro estudio en los $\alpha$-corte de los conjuntos difusos.

\section{Números difusos}
Para poder trabajar con sistemas de ecuaciones diferenciales difusos, es interesante introducir el concepto de números difusos, que podrían ser valores iniciales de nuestro sistema, o números que aparecen directamente en nuestro sistema de ecuaciones diferenciales. \\
Necesitamos en primer lugar dos definiciones antes de definir el concepto de número difuso.

\begin{definicion}[Conjunto difuso normal]
	Un conjunto difuso $A$ decimos que es normal si $nucleo ~ A \neq \emptyset$
\end{definicion}

\begin{definicion}[Conjunto difuso convexo]
	Dado un conjunto difuso $A$ decimos que es convexo si su función de pertenencia es cuasicóncava, esto es:
	\[
		\mu_A(\lambda x + (1-\lambda)y) \geq \min{\{\mu_A(x), \mu_A(y)\}}, \lambda \in [0, 1], x, y \in \mathbb{U}
	\]
\end{definicion}

En este caso, si $\mathbb{U}=\mathbb{R}$ tendríamos que si $A$ es un conjunto difuso convexo, entonces los $\alpha$-corte son intervalos.

Finalmente, introducimos el concepto de número difuso:

\begin{definicion}[Número difuso]
	Decimos que $A$ es un número difuso si $\mathbb{U}=\mathbb{R}$, $A$ es normal y convexo, y además su función de pertencia es continua por la derecha.
\end{definicion}

Observamos que si $x \in \mathbb{R}$ el conjunto $\{x\}$ es un número difuso, con la función de pertenencia.

\subsection{Caracterización números difusos}
A continuación, introducimos dos teoremas que nos permitirán caracterizar los números difusos. Las demostraciones de estos dos resultados se pueden encontrar \cite{apuntesfuzzy}.

\begin{teorema}[Teorema de Stacking]
	Sea $A$ un número difuso, entonces:
	\begin{enumerate}
		\item Sus $\alpha$-cortes son intervalos cerrados no vacíos para todo $\alpha \in [0, 1]$
		\item Si $0 \leq \alpha_1 \leq \alpha_2 \leq 1$ entonces $[A]_{\alpha_1} \subset [A]_{\alpha_2}$
		\item Para toda sucesión no decreciente de $\alpha_n \in [0, 1]$ tal que $a_n \longrightarrow \alpha$ se tiene que:
		\[
			\bigcap^\infty_{n=1} [A]_{\alpha_n} = [A]_\alpha
		\]
		\item Para toda sucesión no creciente $\alpha_n \in [0, 1]$ convergente a $0$ se tiene:
		\[
			cl\left(
				\bigcup^\infty_{n=1} [A]_{\alpha_n} = [A]_0
			\right)
		\]
	\end{enumerate}
\end{teorema}