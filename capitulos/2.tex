\chapter{Cálculo difuso}
En este capítulo exploramos los diferentes acercamientos a \textbf{conceptos clásicos del cálculo mediante una perspectiva difusa}. Exploramos la definición de integral de Riemann, Aumann \& Henstock, y haremos una revisión de la definición de derivada de Hukuhara (\hyperref[def:hukukara]{Diferencia de Hukuhara}). Además, hablaremos de la \textbf{versión difusa del teorema fundamental del cálculo} que nos ayudará a tener una mejor visión de lo que está pasando.

\section{Cálculo difuso para funciones definidas en conjuntos difusos}
En esta sección nos centraremos en aquellas funciones con dominios reales, y que su imagen son conjuntos difusos, es decir,

\begin{definicion}[Funciones definidas en conjuntos difusos]
	\label{def:fizzusetvaluedfunc} Decimos que una función \textbf{$F$ es una función definida en un conjunto difuso} si:
	\begin{enumerate}
		\item $dom ~ F \subset \IR$
		\item $Im ~ F \subset \mathcal{F}_\mathcal{H}(\IR^n)$
	\end{enumerate}
\end{definicion}

\subsection{Derivada}

\subsubsection{La derivada de Hukuhara}
La derivada de Hukuhara está basada en el concepto de \textbf{diferenciabilidad de Hukuhara} para funciones evaluadas en intervalos (\cite{derivatehukuhara})

\begin{definicion}[Diferenciable según Hukuhara]
	Sea $F$ una función definida en un conjunto difuso. Supongamos que los límites:
	
	\[
		\begin{array}{c||c}
			\lim\limits_{h \rightarrow 0^+} \frac{F(x_0 + h) \circleddash_H F(x_0)}{h} & \lim\limits_{h \rightarrow 0^+} \frac{F(x_0) \circleddash_H F(x_0 - h)}{h}
		\end{array}
	\]
	
	Existen, y son iguales a cierto elemento $F'_H(x_0) \in \mathcal{F}_\mathcal{H}(\IR^n)$, entonces $F$ es diferenciable según Hukuhara (H-Diferenciable) en $x_0$ y decimos que $F'_H(x_0)$ es su derivada en $x_0$ 
\end{definicion}

\begin{ejemplo}[Función constante]
	Sea $F(x)=A$ con $A=(-1;0;1)$, calculemos su H-Derivada en un punto arbitrario $x_0 \in \IR$: 
	\[
		F(x_0 + h) \circleddash_H F(x_0) = A \circleddash_H A = 0
	\]
	\[
		F(x_0) \circleddash_H F(x_0 - h) = A \circleddash_H A = 0
	\]
	Por tanto, $F_H'(x_0) = 0$
\end{ejemplo}

\begin{ejemplo}[Función lineal]
	Sea $F(x)=A x$ con $A=(-1; 0; 1)$, supongamos en primer lugar que $x \geq 0$:
	\[
		F(x + h) \circleddash_H F(x) = A(x + h) \circleddash_H A = (-x - h; 0; x+h) \circleddash_H (-x; 0; x) = (-h; 0; h)
	\]
	Análogamente, 
	\[
		F(x) \circleddash_H F(x - h) = (-h; 0; h)
	\]
	De donde,
	\[
	\begin{array}{c||c}
		\lim\limits_{h \rightarrow 0^+} \frac{(-h; 0; h)}{h}=A & \lim\limits_{h \rightarrow 0^+} \frac{(-h; 0; h)}{h}=A
	\end{array}
	\]
	Por tanto, $F_H'(x) = A $, si $x > 0$. \\
	Por otro lado, si $x<0$, podemos ver que $F(x+h) \circleddash_H F(x)$ no está definido, pues:
	
	\begin{itemize}
		\item $(-x-h; 0; x+h)$ no sería un número triangular, pues $-x-h > x+h$
	\end{itemize}

	Esto se puede ver más fácilmente en un resultado que mostramos a continuación.
\end{ejemplo}

\begin{proposicion}[Construcción de funciones H-Diferenciables \cite{spingerfuzzy}]
	Sea $G$ una función definida en conjuntos difusos tal que $G(x)=B g(x)$ donde $g(x)>0, g'(x)>0$ y $B$ es un número difuso entonces, $G(x)$ es H-Diferenciable, más aún, 
	\[
		G'_H(x) = B g'(x)
	\]
\end{proposicion}

Una función H-Diferenciable tiene $\alpha$-corte diferenciables, sin embargo, que todos los $\alpha$-corte sean diferenciables, no implica que la función sea $H$-diferenciable.

\subsubsection{La derivada de Seikkala}
Otro tipo de derivada que veremos, será la derivada de Seikkala, que usaremos después para introducir un concepto de derivada más fuerte.

\begin{definicion}[Derivada de Seikkala]
	Sea $F: [a, b] \rightarrow \mathcal{F}_H(\IR)$ si:
\end{definicion}