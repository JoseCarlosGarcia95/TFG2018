\chapter{Ecuaciones diferenciales difusas}
En este capítulo exploramos los diferentes acercamientos a \textbf{conceptos clásicos del cálculo mediante una perspectiva difusa}. Exploramos la definición de integral de Riemann, Aumann \& Henstock, y haremos una revisión de la definición de derivada de Hukuhara (\hyperref[def:hukukara]{Diferencia de Hukuhara}). Además, hablaremos de la \textbf{versión difusa del teorema fundamental del cálculo} que nos ayudará a tener una mejor visión de lo que está pasando.

\section{Cálculo difuso para funciones definidas en conjuntos difusos}
En esta sección nos centraremos en aquellas funciones con dominios reales, y que su imagen son conjuntos difusos.
\subsection{Derivada}

\subsubsection{La derivada de Hukuhara}
La derivada de Hukuhara está basada en el concepto de \textbf{diferenciabilidad de Hukuhara} para funciones evaluadas en intervalos (\cite{derivatehukuhara})

\begin{definicion}[Diferenciable según Hukuhara]
	Sea $F$ una función definida en un conjunto difuso. Supongamos que los límites:
	
	\[
		\begin{array}{c||c}
			\lim\limits_{h \rightarrow 0^+} \frac{F(x_0 + h) \circleddash_H F(x_0)}{h} & \lim\limits_{h \rightarrow 0^+} \frac{F(x_0) \circleddash_H F(x_0 - h)}{h}
		\end{array}
	\]
	
	Existen, y son iguales a cierto elemento $F'_H(x_0) \in \mathcal{F}_\mathcal{H}(\IR^n)$, entonces $F$ es diferenciable según Hukuhara (H-Diferenciable) en $x_0$ y decimos que $F'_H(x_0)$ es su derivada en $x_0$ 
\end{definicion}

\begin{ejemplo}[Función constante]
	Sea $F(x)=A$ con $A=(-1;0;1)$, calculemos su H-Derivada en un punto arbitrario $x_0 \in \IR$: 
	\[
		F(x_0 + h) \circleddash_H F(x_0) = A \circleddash_H A = 0
	\]
	\[
		F(x_0) \circleddash_H F(x_0 - h) = A \circleddash_H A = 0
	\]
	Por tanto, $F_H'(x_0) = 0$
\end{ejemplo}

\begin{ejemplo}[Función lineal]
	\label{ejemplo:hukuhara}
	Sea $F(x)=A x$ con $A=(-1; 0; 1)$, supongamos en primer lugar que $x \geq 0$:
	\[
		F(x + h) \circleddash_H F(x) = A(x + h) \circleddash_H A = (-x - h; 0; x+h) \circleddash_H (-x; 0; x) = (-h; 0; h)
	\]
	Análogamente, 
	\[
		F(x) \circleddash_H F(x - h) = (-h; 0; h)
	\]
	De donde,
	\[
	\begin{array}{c||c}
		\lim\limits_{h \rightarrow 0^+} \frac{(-h; 0; h)}{h}=A & \lim\limits_{h \rightarrow 0^+} \frac{(-h; 0; h)}{h}=A
	\end{array}
	\]
	Por tanto, $F_H'(x) = A $, si $x > 0$. \\
	Por otro lado, si $x<0$, podemos ver que $F(x+h) \circleddash_H F(x)$ no está definido, pues:
	
	\begin{itemize}
		\item $(-x-h; 0; x+h)$ no sería un número triangular, pues $-x-h > x+h$
	\end{itemize}

	Esto se puede ver más fácilmente en un resultado que mostramos a continuación.
\end{ejemplo}

\begin{proposicion}[Construcción de funciones H-Diferenciables \cite{spingerfuzzy}]
	Sea $G$ una función definida en conjuntos difusos tal que $G(x)=B g(x)$ donde $g(x)>0, g'(x)>0$ y $B$ es un número difuso entonces, $G(x)$ es H-Diferenciable, más aún, 
	\[
		G'_H(x) = B g'(x)
	\]
\end{proposicion}

\begin{teorema}[Continuidad de funciones H-Diferenciables \cite{seikkala}] Sea  $F: [a, b] \rightarrow \mathcal{F}_\mathcal{H}(\IR^n)$ una función H-Diferenciable, entonces $F$ es continua en el sentido visto en el Tema 1.
\end{teorema}

\begin{teorema}[Álgebra en derivadas \cite{seikkala}]
	Sean $F, G : [a, b] \rightarrow \mathcal{F}_\mathcal{H}(\IR^n)$ funciones diferenciables y sea $\lambda \in \IR$. Entonces, $(F+G)_H' = F'_H + G_H'$ y $(\lambda F)_H' = \lambda F'_H$
\end{teorema}

Una función H-Diferenciable tiene $\alpha$-corte diferenciables, sin embargo, que todos los $\alpha$-corte sean diferenciables, no implica que la función sea $H$-diferenciable.

\subsubsection{La derivada de Seikkala}
Otro tipo de derivada que veremos, será la derivada de Seikkala, que usaremos después para introducir un concepto de derivada más fuerte.

\begin{definicion}[Derivada de Seikkala]
	Sea $F: [a, b] \rightarrow \mathcal{F}_H(\IR)$ si:
	\[
		[(f^-_\alpha)'(x_0), (f^+_\alpha)'(x_0)]
	\]
	existe para cada $\alpha \in [0, 1]$ y define $\alpha$-cortes de un número difuso $F'_S(x_0)$ entonces decimos que $F$ es Seikkala diferenciable en $x_0$ y definimos $F'_S(x_0)$ como la derivada de $F$ en $x_0$.
\end{definicion}

A continuación, introducimos algunos teoremas que nos ayudarán a entender mejor este concepto, y nos ayudará a relacionarla con la derivada de Hukuhara:

\begin{teorema}[Condición necesaria Seikkala \cite{seikkala}]
	Si $F: [a, b] \rightarrow \mathcal{F}_\mathcal{H}(\IR^n)$ es H-Diferenciable entonces $f_\alpha^-(x)$ y $f_\alpha^+$ son diferenciables y:
	\[
		[F'(x_0)]_\alpha = [(f_\alpha^-)(x_0), (f_\alpha^+)(x_0)]
	\]
	Entonces, $F$ es diferenciable según Seikkala y la derivada de Seikkala y de Hukuhara coinciden.
\end{teorema}

\subsubsection{La derivada fuertemente generalizada}
El concepto de derivada fuertemente generalizada viene a generalizar más aún los conceptos de derivadas ya introducidos por Hukuhara y Seikkala. \\
Uno de los problemas que nos encontramos con las derivada de Seikkala y Hukuhara es cuando pasa $(f^-_\alpha)'(x_0) \leq (f^+_\alpha)'(x_0)$ (Se puede ver en el Ejemplo \ref{ejemplo:hukuhara}). Esto es que cuando trabajamos con las derivadas de Hukuhara y Seikkala no podemos trabajar con funciones con grado de pertenencia decreciente. Para evitar estos problemas, se introdujo el concepto de derivada fuertemente generalizada:

\begin{definicion}[Derivada fuertemente generalizada]
	Sea $F: (a, b) \rightarrow \mathcal{F}_\mathcal{H}(\IR^n)$. Si alguno de los siguientes pares de límites:
	
	\begin{enumerate}
		\item 	
		\[
		\begin{array}{c||c}
		\lim\limits_{h \rightarrow 0^+} \frac{F(x_0 + h) \circleddash_H F(x_0)}{h} & \lim\limits_{h \rightarrow 0^+} \frac{F(x_0) \circleddash_H F(x_0 - h)}{h}
		\end{array}
		\]
		\item 	
		\[
		\begin{array}{c||c}
		\lim\limits_{h \rightarrow 0^+} \frac{F(x_0 + h) \circleddash_H F(x_0)}{-h} & \lim\limits_{h \rightarrow 0^+} \frac{F(x_0) \circleddash_H F(x_0 - h)}{-h}
		\end{array}
		\]
		\item 	
		\[
		\begin{array}{c||c}
		\lim\limits_{h \rightarrow 0^+} \frac{F(x_0 + h) \circleddash_H F(x_0)}{h} & \lim\limits_{h \rightarrow 0^+} \frac{F(x_0) \circleddash_H F(x_0 - h)}{-h}
		\end{array}
		\]
		\item 	
		\[
		\begin{array}{c||c}
		\lim\limits_{h \rightarrow 0^+} \frac{F(x_0 + h) \circleddash_H F(x_0)}{-h} & \lim\limits_{h \rightarrow 0^+} \frac{F(x_0) \circleddash_H F(x_0 - h)}{h}
		\end{array}
		\]
	\end{enumerate}
	Existen y son iguales a algún elemento $F'_G(x_0)$ de $\mathcal{F}_\mathcal{H}(\IR^n)$, entonces $F$ es diferenciable fuertemente generalizado (o GH) en $x_0$, y $F'_G(x_0)$ es el valor de la derivada. Lo denotamos como GH-Diferenciable.
\end{definicion}

\subsubsection{Otras definiciones}
De la definición es trivial demostrar que si H-Diferenciable entonces es diferenciable fuertemente generalizado, además, la primera condición nos da el concepto de derivada para funciones con diámetro no decreciente. \\
Del mismo modo, la segunda condición nos da un concepto de derivada para funciones con un diámetro no creciente. \\
Los otros casos, conceptualizan el comportamiento cuando varía respecto al diámetro. \\
Por tanto, con la definición anterior la función del Ejemplo \ref{ejemplo:hukuhara} es GH Diferenciable. \\ \\
Para cerrar esta sección, veremos dos conceptos más derivada:

\begin{definicion}[Diferencial de Hukuhara generalizada]
Meter definición
%TODO
\end{definicion}

\begin{definicion}[Diferencial generalizada]
	Meter definición
	%TODO
\end{definicion}

En resumen,
\begin{table}[h]
	\centering
	\begin{tabular}{lllllll}
		H-Diferenciable & $\Rightarrow$ & GH-Diferenciable & $\Rightarrow$ & gH-Diferenciable & $\Rightarrow$ & g-Diferenciable
	\end{tabular}
\end{table}

\subsection{Integral}
La primera propuesta de integral difusa es basada en el concepto integral de Aumann para funciones multievaluadas. La definición original podemos verla en \cite{integral1} y \cite{integral2}. \\
Por simplificar, vamos a denotar:
\[
	S(G) = \{g : I \rightarrow \IR^n : g ~ integrable, g(t) \in G(t), \forall t \in I \}
\]
Con $G : I \rightarrow \mathcal{P}(\IR^n)$

\subsubsection{Integral de Aumann}
\begin{definicion}[Integral de Aumann]
	La integral de Aumann de una función sobre un conjunto difuso $F : [a, b] \rightarrow \mathcal{F}_H(\IR^n)$ sobre $[a, b]$ es definida como: 
	\[
		\left[
		(A) \int_{a}^{b} F(x) dx
		\right]_\alpha = \left\{
			\int_{a}^{b} g(x) dx : g \in S([F(x)]_\alpha)
		\right\}
	\]
	Para todo $\alpha \in [0, 1]$. La función $F$ se dirá que es integrable sobre $[a, b]$ si $(A) \int_{a}^{b} F(x) dx \in \mathcal{F}_\mathcal{H}(\IR^n)$
\end{definicion}

\subsubsection{Integral de Riemann}
Se puede usar también el concepto clásico de integral para definir también una integral de Riemann en el ámbito difuso.

\begin{definicion}[Integral de Riemann]
Meter definición
%TODO
\end{definicion}

\subsubsection{Integral de Henstock}
Tenemos un concepto más fuerte que la integral de Riemann, que nos puede servir para ciertos contextos:
\begin{definicion}[Integral de Henstock]
	Meter definición
	%TODO
\end{definicion}

\subsubsection{Algunos resultados importantes}
En esta sección mostraremos una serie de teoremas de utilidad encontrados en \cite{integral2} que resumen en cierta manera algunas propiedades ya conocidas del cálculo infinitesimal clásico:

\begin{teorema}
	Si una función $F :  [a, b] \rightarrow \mathcal{F}_\mathcal{H}(\IR)$ es continua según (Definición \ref{def:funcioncontinua}) entonces es integrable. Más aún,
	\[
		\left[
			\int F
		\right]_\alpha = \left[
			\int f_\alpha^-, \int f_\alpha^+
		\right]
	\] para todo $\alpha \in [0, 1]$
\end{teorema}

\begin{teorema}[Cambio de intervalos]
	Sea $F :  [a, b] \rightarrow \mathcal{F}_\mathcal{H}(\IR)$ una función integrable, y supongamos $a \leq x_1 \leq x_2 \leq x_3 \leq b$, entonces:
	\[
		\int_{x_1}^{x_3} F = \int_{x_1}^{x_2} F + \int_{x_2}^{x_3} F
	\]
\end{teorema}

\begin{teorema}
	Sean $F, G :  [a, b] \rightarrow \mathcal{F}_\mathcal{H}(\IR)$ funciones integrables, entonces:
	
	\begin{enumerate}
		\item $\int F + G = \int F + \int G$
		\item $\int \lambda F = \lambda \int F$ para cualquier $\lambda \in \IR$.
		\item $d_\infty(F, G)$ es integrable.
		\item $d_\infty(\int F, \int G) \leq \int d_\infty(F, G)$
	\end{enumerate}
\end{teorema}
\subsection{Teorema fundamental del cálculo}