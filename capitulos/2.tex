% !TeX root = ../main.tex

\chapter{Ecuaciones diferenciales difusas}
En este capítulo se van a explorar los diferentes puntos de vista a los \textbf{conceptos clásicos del cálculo mediante una perspectiva difusa}. Se va a recordar la definición de integral de Riemann, Aumann \& Henstock, y haremos una revisión de la definición de derivada de Hukuhara (\hyperref[def:hukukara]{Diferencia de Hukuhara}). Además, se tratará de la \textbf{versión difusa del teorema fundamental del cálculo} que nos ayudará a tener una mejor visión de lo que está pasando.

\section{Cálculo difuso para funciones definidas en conjuntos difusos}
\subsection{Distintas definiciones de derivada}
En primer lugar, vamos a definir los distintos conceptos de derivada difusa.

\subsubsection{La derivada de Hukuhara}
La derivada de Hukuhara está basada en el concepto de \textbf{diferenciabilidad de Hukuhara} para funciones evaluadas en intervalos (\cite{derivatehukuhara})

\begin{definicion}[Diferenciable según Hukuhara]
  Sea $F$ una función definida en un conjunto difuso. Se supone que que los límites:
  
  \[
  \begin{array}{c||c}
    \lim\limits_{h \rightarrow 0^+} \frac{F(x_0 + h) \circleddash_H F(x_0)}{h} & \lim\limits_{h \rightarrow 0^+} \frac{F(x_0) \circleddash_H F(x_0 - h)}{h}
  \end{array}
  \]
  
  Existen, y son iguales a cierto elemento $F'_H(x_0) \in \mathcal{F}_\mathcal{H}(\IR^n)$, entonces $F$ \textbf{es diferenciable según Hukuhara (H-Diferenciable)} en $x_0$ y se dice que $F'_H(x_0)$ es su derivada en $x_0$ 
\end{definicion}

\begin{ejemplo}[Función constante]
  Sea $F(x)=A$ con $A=(-1;0;1)$, se va a calcular su H-Derivada en un punto arbitrario $x_0 \in \IR$: 
  \[
  F(x_0 + h) \circleddash_H F(x_0) = A \circleddash_H A = 0
  \]
  \[
  F(x_0) \circleddash_H F(x_0 - h) = A \circleddash_H A = 0
  \]
  Por tanto, $F_H'(x_0) = 0$
\end{ejemplo}

\begin{ejemplo}[Función lineal]
  \label{ejemplo:hukuhara}
  Sea $F(x)=A x$ con $A=(-1; 0; 1)$, se supone en primer lugar que $x \geq 0$:
  \[
  F(x + h) \circleddash_H F(x) = A(x + h) \circleddash_H A = (-x - h; 0; x+h) \circleddash_H (-x; 0; x) = (-h; 0; h)
  \]
  Análogamente, 
  \[
  F(x) \circleddash_H F(x - h) = (-h; 0; h)
  \]
  De donde,
  \[
  \begin{array}{c||c}
    \lim\limits_{h \rightarrow 0^+} \frac{(-h; 0; h)}{h}=A & \lim\limits_{h \rightarrow 0^+} \frac{(-h; 0; h)}{h}=A
  \end{array}
  \]
  Por tanto, $F_H'(x) = A $, si $x > 0$. \\
  Por otro lado, si $x<0$, se puede ver que $F(x+h) \circleddash_H F(x)$ no está definido, pues:
  
  \begin{itemize}
  \item $(-x-h; 0; x+h)$ no sería un número triangular, pues $-x-h > x+h$
  \end{itemize}

  Esto se puede ver más fácilmente en un resultado que se muestra a continuación.
\end{ejemplo}

\begin{proposicion}[Construcción de funciones H-Diferenciables \cite{spingerfuzzy}]
  Sea $G$ una función definida en conjuntos difusos tal que $G(x)=B g(x)$ donde $g(x)>0, g'(x)>0$ y $B$ es un número difuso entonces, $G(x)$ es H-Diferenciable, más aún, 
  \[
  G'_H(x) = B g'(x)
  \]
\end{proposicion}

\begin{teorema}[Continuidad de funciones H-Diferenciables \cite{seikkala}] Sea  $F: [a, b] \rightarrow \mathcal{F}_\mathcal{H}(\IR^n)$ una función H-Diferenciable, entonces $F$ es continua en el sentido visto en el Tema 1.
\end{teorema}

\begin{teorema}[Álgebra en derivadas \cite{seikkala}]
  Sean $F, G : [a, b] \rightarrow \mathcal{F}_\mathcal{H}(\IR^n)$ funciones diferenciables y sea $\lambda \in \IR$. Entonces, $(F+G)_H' = F'_H + G_H'$ y $(\lambda F)_H' = \lambda F'_H$
\end{teorema}

\textbf{Una función H-Diferenciable tiene $\alpha$-corte diferenciables}, sin embargo, \textbf{que todos los $\alpha$-corte sean diferenciables, no implica que la función sea $H$-diferenciable.}

\subsubsection{La derivada de Seikkala}
Ahora se introduce el concepto de derivada de Seikkala, que se usará después para \textbf{introducir un concepto de derivada más fuerte.}

\begin{definicion}[Derivada de Seikkala]
  Sea $F: [a, b] \rightarrow \mathcal{F}_H(\IR)$ si:
  \[
    [(f^-_\alpha)'(x_0), (f^+_\alpha)'(x_0)]
    \]
    existe para cada $\alpha \in [0, 1]$ y define $\alpha$-cortes de un número difuso $F'_S(x_0)$ entonces se dice que $F$ es Seikkala diferenciable en $x_0$ y definimos $F'_S(x_0)$ como la derivada de $F$ en $x_0$.
\end{definicion}

A continuación, se introduce una condición necesaria que servirá de herramienta para relacionar el concepto anteriormente visto de derivada y la derivada de Seikkala.

\begin{teorema}[Condición necesaria Seikkala \cite{seikkala}]
  Si $F: [a, b] \rightarrow \mathcal{F}_\mathcal{H}(\IR^n)$ es H-Diferenciable entonces $f_\alpha^-(x)$ y $f_\alpha^+$ son diferenciables y:
  \[
    [F'(x_0)]_\alpha = [(f_\alpha^-)(x_0), (f_\alpha^+)(x_0)]
    \]
    Entonces, $F$ es diferenciable según Seikkala y la derivada de Seikkala y de Hukuhara coinciden.
\end{teorema}

\subsubsection{La derivada fuertemente generalizada}
El concepto de derivada fuertemente generalizada \textbf{viene a generalizar más aún los conceptos de derivadas ya introducidos por Hukuhara y Seikkala.} \\

Uno de los problemas que nos encontramos con las derivada de Seikkala y Hukuhara es cuando pasa $(f^-_\alpha)'(x_0) \leq (f^+_\alpha)'(x_0)$ (Se puede ver en el Ejemplo \ref{ejemplo:hukuhara}). Para evitar estos problemas, se introdujo el concepto de derivada fuertemente generalizada:

\begin{definicion}[Derivada fuertemente generalizada]
  Sea $F: (a, b) \rightarrow \mathcal{F}_\mathcal{H}(\IR^n)$. Si alguno de los siguientes pares de límites:
  
  \begin{enumerate}
  \item 	
    \[
    \begin{array}{c||c}
      \lim\limits_{h \rightarrow 0^+} \frac{F(x_0 + h) \circleddash_H F(x_0)}{h} & \lim\limits_{h \rightarrow 0^+} \frac{F(x_0) \circleddash_H F(x_0 - h)}{h}
    \end{array}
    \]
  \item 	
    \[
    \begin{array}{c||c}
      \lim\limits_{h \rightarrow 0^+} \frac{F(x_0 + h) \circleddash_H F(x_0)}{-h} & \lim\limits_{h \rightarrow 0^+} \frac{F(x_0) \circleddash_H F(x_0 - h)}{-h}
    \end{array}
    \]
  \item 	
    \[
    \begin{array}{c||c}
      \lim\limits_{h \rightarrow 0^+} \frac{F(x_0 + h) \circleddash_H F(x_0)}{h} & \lim\limits_{h \rightarrow 0^+} \frac{F(x_0) \circleddash_H F(x_0 - h)}{-h}
    \end{array}
    \]
  \item 	
    \[
    \begin{array}{c||c}
      \lim\limits_{h \rightarrow 0^+} \frac{F(x_0 + h) \circleddash_H F(x_0)}{-h} & \lim\limits_{h \rightarrow 0^+} \frac{F(x_0) \circleddash_H F(x_0 - h)}{h}
    \end{array}
    \]
  \end{enumerate}
  Existen y son iguales a algún elemento $F'_G(x_0)$ de $\mathcal{F}_\mathcal{H}(\IR^n)$, entonces $F$ es diferenciable fuertemente generalizado (o GH) en $x_0$, y $F'_G(x_0)$ es el valor de la derivada. Se denota como GH-Diferenciable.
\end{definicion}
De la definición \textbf{es trivial demostrar que si H-Diferenciable entonces es diferenciable fuertemente generalizado}.

\begin{observacion}
  De la definición anterior se pueden extraer las siguientes conclusiones, basadas en la naturaleza del diámetro de las funciones difusas:
  \begin{enumerate}
  \item Concepto de derivada para funciones con diámetro no decreciente.
  \item Concepto de derivada para funciones con diámetro no creciente.
  \item Concepto de derivada para funciones con diámetro con monotonía arbitraria.
  \item Concepto de derivada para funciones con diámetro con monotonía arbitraria.
  \end{enumerate}
\end{observacion} 

\subsubsection{Otras definiciones}

Para cerrar esta sección, veremos dos conceptos más derivada:

\begin{definicion}[Diferencial de Hukuhara generalizada]
  Sea $F: (a, b) \longrightarrow \mathcal{F}_\mathcal{C}(\IR)$. Si el límite:
  \[
  	\lim\limits_{h \rightarrow 0} \frac{F(x_0 + h) \circleddash_{gH} F(x_0)}{h}
  \]
  existe y pertenece a $\mathcal{F}_\mathcal{C} (\IR)$, entonces $F$ es diferenciable de forma generalizada según Hukuhara (gH diferenciable) en $x_0$, además a $F'_{gH}(x_0)$ se le llama el valor de la derivada de forma generalizada según Hukukahara
\end{definicion}

\begin{definicion}[Diferencial generalizada]
  \label{def:diferencial_generalizada}
  Sea $F: (a, b) \longrightarrow \mathcal{F}_\mathcal{C}(\IR)$. Si el límite:
\[
\lim\limits_{h \rightarrow 0} \frac{F(x_0 + h) \circleddash_{g} F(x_0)}{h}
\]
existe y pertenece a $\mathcal{F}_\mathcal{C} (\IR)$, entonces $F$ es diferenciable de forma generalizada (g diferenciable) en $x_0$, además a $F'_{g}(x_0)$ se le llama el valor de la derivada de forma generalizada
\end{definicion}

En resumen,
\begin{table}[h]
  \centering
  \begin{tabular}{lllllll}
    H-Diferenciable & $\Rightarrow$ & GH-Diferenciable & $\Rightarrow$ & gH-Diferenciable & $\Rightarrow$ & g-Diferenciable
  \end{tabular}
\end{table}

\subsection{Integral}
La primera propuesta de integral difusa es \textbf{basada en el concepto integral de Aumann \cite{aumannintegral}} para funciones multievaluadas. La definición original podemos verla en \cite{integral1} y \cite{integral2}. \\
Por simplificar, vamos a denotar:
\[
S(G) = \{g : I \rightarrow \IR^n : g ~ integrable, g(t) \in G(t), \forall t \in I \}
\]
Con $G : I \rightarrow \mathcal{P}(\IR^n)$

\subsubsection{Integral de Aumann}
\begin{definicion}[Integral de Aumann]
  La integral de Aumann de una función sobre un conjunto difuso $F : [a, b] \rightarrow \mathcal{F}_H(\IR^n)$ sobre $[a, b]$ es definida como: 
  \[
  \left[
    (A) \int_{a}^{b} F(x) dx
    \right]_\alpha = \left\{
  \int_{a}^{b} g(x) dx : g \in S([F(x)]_\alpha)
  \right\}
  \]
  Para todo $\alpha \in [0, 1]$. La función $F$ se dirá que es integrable sobre $[a, b]$ si $(A) \int_{a}^{b} F(x) dx \in \mathcal{F}_\mathcal{H}(\IR^n)$
\end{definicion}

\subsubsection{Integral de Riemann}
Se puede usar también el concepto clásico de integral para definir también una \textbf{integral de Riemann en el ámbito difuso.}

\begin{definicion}[Integral de Riemann]
  La integral de Riemman de una función difusa $F : [a, b] \rightarrow \mathcal{F}_\mathcal{C} (\IR)$ sobre $[a, b]$ es el número difuso $A$ tal que para todo $\varepsilon > 0$ existe un $\delta > 0$ tal que para cualquier partición $\mathcal{P}: = a=x_0 < x1 < ... < x_n = b$ con $x_i - x_{i-1} < \delta, i = 1, ..., n$ y $\xi_i \in [x_i - x_{i-1}]$
  \[
  	d_\infty \left(
  		\sum_{i=1}^{n-1} F(\xi_i)(x_i - x_{i-1}, A
  	\right) < \varepsilon
  \]
  La función $F$ se dice que es integrable según Riemann sobre $[a, b]$ y se denota:
  \[
  	(R) \int_{a}^{b} F(x) dx = A
  \]
\end{definicion}

\subsubsection{Algunos resultados importantes}
En esta sección mostraremos una serie de teoremas de utilidad encontrados en \cite{integral2} que resumen en cierta manera \textbf{algunas propiedades ya conocidas del cálculo infinitesimal clásico}:

\begin{teorema}
  Si una función $F :  [a, b] \rightarrow \mathcal{F}_\mathcal{H}(\IR)$ es continua según (Definición \ref{def:funcioncontinua}) entonces es integrable. Más aún,
  \[
  \left[
    \int F
    \right]_\alpha = \left[
    \int f_\alpha^-, \int f_\alpha^+
    \right]
  \] para todo $\alpha \in [0, 1]$
\end{teorema}

\begin{teorema}[Cambio de intervalos]
  Sea $F :  [a, b] \rightarrow \mathcal{F}_\mathcal{H}(\IR)$ una función integrable, y supongamos $a \leq x_1 \leq x_2 \leq x_3 \leq b$, entonces:
  \[
  \int_{x_1}^{x_3} F = \int_{x_1}^{x_2} F + \int_{x_2}^{x_3} F
  \]
\end{teorema}

\begin{teorema}
  Sean $F, G :  [a, b] \rightarrow \mathcal{F}_\mathcal{H}(\IR)$ funciones integrables, entonces:
  
  \begin{enumerate}
  \item $\int F + G = \int F + \int G$
  \item $\int \lambda F = \lambda \int F$ para cualquier $\lambda \in \IR$.
  \item $d_\infty(F, G)$ es integrable.
  \item $d_\infty(\int F, \int G) \leq \int d_\infty(F, G)$
  \end{enumerate}
\end{teorema}

\subsection{Teorema fundamental del cálculo}
En la teoría clásica del análisis, \textbf{el teorema fundamental del cálculo nos ofrece una visión bastante fuerte que relaciona las derivadas y las integrales}. En el cálculo difuso tenemos lo mismo, y los siguientes teoremas que vamos a introducir nos \textbf{generalizarán estos teoremas clásicos para casos difusos.}

\begin{teorema}[Teorema fundamental del cálculo difuso\cite{integral2}]
  Sea $F : [a, b] \rightarrow \mathcal{F}_\mathcal{H} (\IR^n)$ una función continua, entonces $G(x) = \int_{a}^{x} F(s) ds$ es H-Diferenciable y además,
  \[
  G'_H(x) = F(x)
  \]
\end{teorema}

\begin{teorema}[\cite{integral2}]
  Sea $F : [a, b] \rightarrow  \mathcal{F}_\mathcal{H} (\IR^n)$ una función H-Diferenciable y sea $F_H'$ su derivada integrable sobre $[a, b]$. Entonces,
  \[
  F(x) = F(a) + \int_{a}^{x}F_H'(s) ds
  \]
  Para todo $x \in [a, b]$
\end{teorema}

Hay un teorema parecido al anterior, pero esta vez exigiendo que la función es diferenciable fuertemente generalizada (2), que dice lo siguiente:

\begin{teorema}[\cite{fundamental}]
  Sea $F : [a, b] \rightarrow  \mathcal{F}_\mathcal{H} (\IR^n)$ una función diferenciable fuertemente generalizada (2) y sea $F_H'$ su derivada integrable sobre $[a, b]$. Entonces,
  \[
  F(x) = F(a) + \int_{a}^{x}F_H'(s) ds
  \]
  Para todo $x \in [a, b]$
\end{teorema}

\section{Ecuaciones diferenciales difusas}
Del mismo modo que \textbf{hay varios acercamientos al concepto de derivada}, no podía ser menos el concepto de ecuación diferencial difusa. En esta sección \textbf{vamos a tratar los distintas formas de acercarse a los conceptos de ecuación diferencial difusa}. Esta sección toma como referencia \cite{fuzzyapproaches}

\subsection{Introducción}
En esta sección vamos a empezar planteando un problema difuso de la misma manera que se introducen las ecuaciones diferenciales clásicas:

\begin{equation}
  \label{def:edf}
  X'(t) = f(t, X(t)), ~ X(0) = X_0
\end{equation}
Donde $f : [O, T] \times \mathcal{F}_\mathcal{H}(\IU) \rightarrow \mathcal{F}(\IR^n)$ se obtiene mediante el \hyperref[def:zadeh]{principio de extensión de Zadeh} aplicada a una función continua $g : [0, T] \times U \rightarrow \IR^n$. Además, $f$  es continua por ser $g$ continua, aplicando \hyperref[teorema:contfuzzy]{Teorema \ref*{teorema:contfuzzy}} se puede concluir:
\[
	[f(t, X)]_\alpha = g(t, [X]_\alpha)
\]
Se puede \textbf{asociar a esta ecuación diferencial difusa una ecuación diferencial ordinaria:}
\begin{equation}
	\label{eq:edo}
	x'(t) = g(t, x(t)), ~ x(0) = c
\end{equation}

\subsection{Teorema de equivalencia entre EDO y EDD}

A continuación, se ofrece un teorema que dará una relación entre ecuaciones diferenciales difusas y ordinarias.

\begin{teorema}[Equivalencia entre EDO y EDD \cite{fuzzyapproaches1}]
	\label{teorema:equivalencia}
	Sea $U$ un conjunto abierto sobre $\IR^n$ y sea $[X_0]_\alpha \subset U$ con $\alpha \in [0, 1]$. Sea $g$ una función continua, y suponga que para cada $c \in U$ existe una única solución $x(\cdot, c)$ del problema \ref{eq:edo} y también que $x(t, \cdot)$ es continúa para todo $t \in [0, T]$ fijado. Entonces, existe una única solución difusa $X(t) = \hat{x}(t, x_0)$ del problema \ref{def:edf}
\end{teorema}
El resultado anterior es muy potente, \textbf{permitirá utilizar métodos numéricos ya conocidos para resolver ecuaciones diferenciales difusas fácilmente.}

Para ver la potencia del Teorema anterior, se introducirán una serie de ejemplos:

\begin{ejemplo}
	Se plantea la siguiente ecuación diferencial difusa con valores iniciales:
	\[
		\left\{
			\begin{array}{ccc}
				X' & = & - X(t) \\
				X(0) & =  & C
			\end{array}
		\right.
	\]
	
	Donde $C$ representa un número difuso arbitrario. A continuación se construye el problema ordinario asociado al problema difuso de la misma manera que se hace al inicio de esta sección;
	
	\[
		\left\{
			\begin{array}{ccc}
				x' & = & - x(t) \\
				x(0) & =  & c
			\end{array}
		\right.
	\]
	
	Es bien conocido que este problema de valores iniciales tiene la siguiente solución:
	\[
		x(t, c) = c e^{-t}
	\]
	
	Por tanto, dado que $x(t, c)$ es continúa para cada $t, c \in \IR$ se puede aplicar el \hyperref[teorema:equivalencia]{Teorema de equivalencia} y tenemos que la solución del problema difuso es:
	\[
		X(t) = C \cdot e^{-t}
	\]
\end{ejemplo}

\begin{ejemplo}
	Se plantea ahora un problema un tanto más interesante:
	\[
		\left\{
			\begin{array}{ccc}
				X' & = & X^2(t) \\
				X & =  & C
			\end{array}
		\right.
	\]
	
	Donde esta vez $C$ es un número difuso triangular:
	\[
		C(y) = \left\{
			\begin{array}{ccc}
				3 - y & si & 2 \leq y \leq 3 \\
				y - 1 & si & 1 \leq y \leq 2 \\
				0 
			\end{array}
		\right.
	\]
	El problema ordinario (o determinístico asociado) viene dado por:
	\[
		x'(t) = x^2(t), ~ x(0) = c
	\]
	y la solución de este problema viene dado por:
	\[
		x(t, c) = \frac{c}{1-tc}
	\]
	Para cada $t \in [0, \frac{1}{3})$ fijado, la función $(x, t)$ es continua respecto a $c$, entonces se puede aplicar el \hyperref[teorema:equivalencia]{Teorema de equivalencia} y podemos concluir que existe una única solución difusa dada por $X(t) = \hat{x}(t, X_0)$. Se observa que es posible calcular la solución difusa aplicando directamente el principio de extensión de Zadeh y el \hyperref[teorema:contfuzzy]{Teorema \ref*{teorema:contfuzzy}}. Por tanto, se pueden calcular los $\alpha$-cortes de la siguiente forma para cada $\alpha \in [0, 1]$
	
	\[
		\begin{array}{ccc}
			[X(t)]_\alpha & = & [\hat{x}(t, X_0)]_\alpha \\
			& = & x(t, [X_0]_\alpha) \\
			& = & x(t, [1+\alpha, 3-\alpha]) \\
			& = & [x(t, 1+\alpha), x(t, 3-\alpha)] \\
			& = & \left[
				\frac{1+\alpha}{1-t-t\alpha}, \frac{3-\alpha}{1-3t + t\alpha}
			\right]
		\end{array}
	\]
\end{ejemplo}

\subsection{Inclusiones diferenciales}
Esta sección está basada en las observaciones que se pueden ver en los artículos \cite{inclusionesdif1}, \cite{inclusionesdif2} y \cite{inclusionesdif3}. \\

En estos artículos Hüllermeier y Diamond interpretan las  \hyperref[def:edf]{Ecuaciones Diferenciales Difusas} como una familia inclusiones diferenciales:

\begin{equation}
	\label{def:familiainclusionesdif}
	y'_\alpha(t) = g(t, y_\alpha(t)), ~ y_\alpha(0) \in [X_0]_\alpha, ~ 0 \leq \alpha \leq 1
\end{equation}

Bajo unas determinadas hipótesis, podemos afirmar:
\[
	\mathcal{A}_\alpha = \{y_\alpha | y_\alpha ~ \text{es una solución de la \hyperref[def:familiainclusionesdif]{Familia de inclusiones diferenciales}}\}
\]
Son los $\alpha$-corte de un determinado conjunto difuso y se le llaman soluciones del problema dado por la \hyperref[def:familiainclusionesdif]{Familia de inclusiones diferenciales}

A continuación se ofrece un teorema análogo al teorema que se vio en la sección introductoria \hyperref[teorema:equivalencia]{Teorema de Equivalencia entre EDO y EDF}
\begin{teorema}[\cite{fuzzyapproaches1}]
	Sea $U$ un conjunto abierto en $\IR^n$ y sea $X_0 \in \mathcal{F}(U)$. Dada la función $g$ continua, para cada $c \in U$ existe una única solución $x(\cdot, c)$ del problema \ref{eq:edo} y que $x(t, \cdot)$ es continúa en $U$ para cada $t \in [0, T]$. Entonces, la solución difusa del problema \ref{def:edf} y las soluciones del problema de \hyperref[def:familiainclusionesdif]{Familia de inclusiones diferenciales} coinciden, es decir;
	\[
		X(t) = \mathcal{A}(t)
	\]
	Para todo $t \in [0, T]$
\end{teorema}
