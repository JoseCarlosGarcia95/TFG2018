% !TeX root = ../main.tex
\begin{abstract}
	Este trabajo trata de introducir la teoría de lógica difusa, y buscar nuevos conceptos relacionados con la concepción de una teoría de análisis difuso para finalmente llegar al concepto de ecuación diferencial difusa. Los pilares fundamentales donde se va a basar esta teoría subyacen en el poder de la definición del principio de extensión de Zadeh, los conceptos de números difusos, y las diferentes operaciones aritméticas que se pueden definir. \\ \\
	Mediante unas condiciones de regularidad no demasiado restrictivas, se puede encontrar una relación entre resolver una ecuación diferencial difusa y una ecuación diferencial determinista, este marco teórico, invita y acompaña a construir métodos numéricos basados en los métodos numéricos clásicos que ya son conocidos por todo el mundo. \\ \\
	Este trabajo, también introduce algunos modelos que se pueden construir teniendo en cuenta la naturaleza difusa de la vida real. \\ \\
	Finalmente, en esta memoria también se trabajan técnicas computacionales de alto rendimiento, para conseguir resolver los problemas de forma eficiente en cuestión de tiempos y de energía. Estas observaciones nos mostrará que las técnicas computacionales de alto rendimiento pueden cambiar de forma notable los resultados de las pruebas.
\end{abstract}