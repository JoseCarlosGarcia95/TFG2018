% Información
\title{Resolución numérica de ecuaciones diferenciales difusas}
\university{Universidad de Cádiz}
\degree{Grado en Matemáticas}
\type{Trabajo fin de grado}
\subdate{Curso académico: 2017/2018}
\author{José Carlos García Ortega}
\advisor{Dr. Rafael Rodríguez Galván}
\advisortwo{Dr. Jesús Medina Moreno}
\city{Jerez de la Frontera, Cádiz}
\signaturecandidate{Firma del alumno}
\signatureadvisor{Firma del tutor}
\signatureadvisortwo{Firma del tutor}

% Teoremas
\newtheorem{teorema}{Teorema}
\newtheorem{proposicion}{Proposición}
\newtheorem{definicion}{Definición}
\newtheorem{observacion}{Observación}
\newtheorem{ejemplo}{Ejemplo}

% Fuentes
\usepackage[math]{kurier}
\usepackage[sfdefault,condensed]{roboto}
\usepackage[T1]{fontenc}

% Macros
\def\IR{\ensuremath{\mathbb R}}
\def\IK{\ensuremath{\mathbb K}}
\def\IC{\ensuremath{\mathbb C}}
\def\IN{\ensuremath{\mathbb N}}
\def\IZ{\ensuremath{\mathbb Z}}
\def\IQ{\ensuremath{\mathbb Q}}
\def\IU{\ensuremath{\mathbb U}}
\def\FUZZYM{\ensuremath{\mu_\mathcal{A}}}
\def\HDIFF{\ensuremath{\circleddash_H}}
% Titulos
\rhead{}

% Hacer enlaces clickables.
\usepackage{hyperref}
\hypersetup{
	colorlinks=true,
	linkcolor=blue,
	citecolor=blue,
	filecolor=magenta,
	urlcolor=cyan
}

% Notación
\usepackage{tabulary}
\usepackage{booktabs}
\usepackage{mathtools}  

% Copyright
\usepackage{fix-cm}
\makeatletter
\DeclareRobustCommand{\regmark}{\raisebox{1.13ex}{%
		\fontsize{.6\dimexpr\f@size pt}\z@\selectfont\textregistered}%
}
\makeatother

% Código
\usepackage{listings}
\usepackage{booktabs}

% Añadimos subsubsection al indice
%\setcounter{tocdepth}{3}
%\setcounter{secnumdepth}{3}

% Subfigure
\usepackage{subcaption}
\usepackage{float}

\makeatletter
\renewcommand{\@chapapp}{}% Not necessary...
\newenvironment{chapquote}[2][2em]
{\setlength{\@tempdima}{#1}%
	\def\chapquote@author{#2}%
	\parshape 1 \@tempdima \dimexpr\textwidth-2\@tempdima\relax%
	\itshape}
{\par\normalfont\hfill--\ \chapquote@author\hspace*{\@tempdima}\par\bigskip}
\makeatother